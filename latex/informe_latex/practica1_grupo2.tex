%package list
\documentclass{article}
\usepackage[top=3cm, bottom=3cm, outer=3cm, inner=3cm]{geometry}
\usepackage{graphicx}
\usepackage{url}
%\usepackage{cite}
\usepackage{hyperref}
\usepackage{array}
\usepackage{multicol}
\newcolumntype{x}[1]{>{\centering\arraybackslash\hspace{0pt}}p{#1}}
\usepackage{natbib}
\usepackage{pdfpages}
\usepackage{multirow}
\usepackage{float}
\usepackage[normalem]{ulem}
\useunder{\uline}{\ul}{}


%%%%%%%%%%%%%%%%%%%%%%%%%%%%%%%%%%%%%%%%%%%%%%%%%%%%%%%%%%%%%%%%%%%%%%%%%%%%
%%%%%%%%%%%%%%%%%%%%%%%%%%%%%%%%%%%%%%%%%%%%%%%%%%%%%%%%%%%%%%%%%%%%%%%%%%%%
\newcommand{\csemail}{vmachacaa@unsa.edu.pe}
\newcommand{\csdocente}{Vicente Machaca Arceda}
\newcommand{\cscurso}{Algoritmos y Estructura de Datos}
\newcommand{\csuniversidad}{Universidad Nacional de San Agustín}
\newcommand{\csescuela}{Maestría en Ciencia de la Computación}
\newcommand{\cspracnr}{01}
\newcommand{\cstema}{--}
%%%%%%%%%%%%%%%%%%%%%%%%%%%%%%%%%%%%%%%%%%%%%%%%%%%%%%%%%%%%%%%%%%%%%%%%%%%%
%%%%%%%%%%%%%%%%%%%%%%%%%%%%%%%%%%%%%%%%%%%%%%%%%%%%%%%%%%%%%%%%%%%%%%%%%%%%


\usepackage[english,spanish]{babel}
\usepackage[utf8]{inputenc}
\AtBeginDocument{\selectlanguage{spanish}}
\renewcommand{\figurename}{Figura}
\renewcommand{\refname}{Referencias}
\renewcommand{\tablename}{Tabla} %esto no funciona cuando se usa babel
\AtBeginDocument{%
	\renewcommand\tablename{Tabla}
}

\usepackage{fancyhdr}
\pagestyle{fancy}
\fancyhf{}
\setlength{\headheight}{30pt}
\renewcommand{\headrulewidth}{1pt}
\renewcommand{\footrulewidth}{1pt}
\fancyhead[L]{\raisebox{-0.2\height}{\includegraphics[width=3cm]{img/logo_unsa}}}
\fancyhead[C]{}
\fancyhead[R]{\fontsize{7}{7}\selectfont	\csuniversidad \\ \csescuela \\ \textbf{\cscurso} }
\fancyfoot[L]{MSc. Vicente Machaca}
\fancyfoot[C]{\cscurso}
\fancyfoot[R]{Página \thepage}


\begin{document}
	
	\vspace*{10px}
	
	\begin{center}	
		\fontsize{17}{17} \textbf{ Práctica \cspracnr}
	\end{center}
	%\centerline{\textbf{\underline{\Large Título: Informe de revisión del estado del arte}}}
	%\vspace*{0.5cm}
	

	\begin{table}[h]
		\begin{tabular}{|x{4.7cm}|x{4.8cm}|x{4.8cm}|}
			\hline
			\textbf{DOCENTE} & \textbf{CARRERA}  & \textbf{CURSO}   \\
			\hline
			\csdocente & \csescuela & \cscurso    \\
			\hline
		\end{tabular}
	\end{table}	
	
	
	\begin{table}[h]
		\begin{tabular}{|x{4.7cm}|x{4.8cm}|x{4.8cm}|}
			\hline
			\textbf{PRÁCTICA} & \textbf{TEMA}  & \textbf{DURACIÓN}   \\
			\hline
			\cspracnr & \cstema & 3 horas   \\
			\hline
		\end{tabular}
	\end{table}
	
	
	\section{Datos de los estudiantes}
	\begin{itemize}
		\item Grupo: 2
		\item Integrantes:
		\begin{itemize}
			\item EDER ALONSO AMPUERO ATAMARI
			\item HOWARD FERNANDO ARANZAMENDI MORALES
            \item JOSE EDISON PEREZ MAMANI
            \item HENRRY IVAN ARIAS MAMANI
		\end{itemize}		
	\end{itemize}
	\section{Algoritmo de Ordenamiento}\label{sec:ejercicios}
    \subsection{caracteristicas}
     \subsection{MergeSort}
    \paragraph{}
    El Merge Sort es un algoritmo recursivo bastante eficiente para ordenar un array, que tiene un orden de complejidad O(nlogn) al igual que Quick Sort. fue desarrollado en 1945 por John Von Neumann.

    El Merge Sort está basado en la técnica de diseño de algoritmos Divide y Vencerás, esta técnica consiste en dividir el problema a resolver en sub problemas del mismo tipo que a su vez se dividirán, mientras no sean suficientemente  pequeños o triviales.
    \begin{figure}[h!]
        \centering
        \includegraphics[width=12cm]{img/mergesort.png}
        \caption{Estrategia que sigue algoritmo para ordenar una secuencia S de n elementos}
        \label{fig:mergesort}
    \end {figure}
    \begin{itemize}
        \item Si S tiene uno o ningún elemento, está ordenada.
        \item Si S tiene al menos dos elementos se divide en dos secuencias S1 y S2.
        \item S1 contiene los primeros n/2 elementos y S2 los restantes.
        \item Ordenar S1 y S2, aplicando recursivamente este procedimiento
        \item Mezclar S1 y S2 en S, de forma que ya S1 y S2 estén ordenados
        \item Veamos ahora como sería la estrategia para mezclar las secuencias:
    \end{itemize}
    \paragraph {}
    Se tienen referencias al principio de cada una de las secuencias a mezclar (S1 y S2). Mientras en alguna secuencia queden elementos, se inserta en la secuencia resultante (S) el menor de los elementos referenciados y se avanza esa referencia una posición.
      \subsubsection{Gráfica MergeSort}
        \begin{figure}[h!]
            \centering
            \includegraphics[width=12cm]{img/mergeSort_1.png}
            \caption{Estrategia que sigue algoritmo para ordenar una secuencia S de n elementos}
            \label{fig:mergesort}
        \end {figure}
        \begin{figure}[h!]
            \centering
            \includegraphics[width=12cm]{img/mergeSort_2.png}
            \caption{Estrategia que sigue algoritmo para ordenar una secuencia S de n elementos}
            \label{fig:mergesort}
        \end {figure}
    \begin{table}[]
        \begin{tabular}{|c|c|c|c|c|c|c|c| }
            \hline
            \multicolumn{8}{|c|}{Algoritmo: Merge Sort} \\ \hline
            \multicolumn{4}{|c|}{} & \multicolumn{4}{c|}{Lenguaje: C++} \\ \hline
              N de Datos &     t1    &  t2         &  t3          &   t4        &    t5     &   Promedio(t)       & desv. s. \\ \hline
100	    &0.1487	&0.0874	&0.1071	&0.0755	&0.11	&0.10574	&0.027912237\\ \hline
1000	&1.2993	&0.8411	&0.9458	&1.2623	&1.3617	&1.14204	&0.232657856\\ \hline
2000	&1.743	&1.5607	&2.0408	&2.788	&2.0492	&2.03634	&0.468364119\\ \hline
3000	&3.1413	&3.3076	&3.7817	&3.0035	&2.7599	&3.1988	&0.382652388\\ \hline
4000	&3.5795	&3.8921	&3.898	&5.6032	&6.5539	&4.70534	&1.304223862\\ \hline
5000	&7.2039	&4.8042	&5.7385	&8.846	&6.0797	&6.53446	&1.55125157\\ \hline
6000	&7.7235	&5.8136	&8.4027	&5.9593	&9.3577	&7.45136	&1.542865924\\ \hline
7000	&7.6291	&7.3039	&8.143	&7.7513	&7.0229	&7.57004	&0.428607954\\ \hline
8000	&10.2886	&8.213	&10.2271	&11.791	&8.7372	&9.85138	&1.415992469\\ \hline
9000	&9.2446	&18.651	&13.6912	&9.715	&10.9793	&12.45622	&3.87009553\\ \hline
10000	&11.0627	&9.9283	&11.7518	&11.2358	&12.4952	&11.29476	&0.945315134\\ \hline
20000	&21.3784	&22.2088	&21.4662	&23.3717	&21.1154	&21.9081	&0.913342274\\ \hline
30000	&33.4084	&32.9699	&44.6192	&35.4287	&36.7651	&36.63826	&4.71867004\\ \hline
40000	&48.641	&44.7479	&46.0825	&42.4689	&41.8431	&44.75668	&2.76444856\\ \hline
50000	&65.7168	&54.3054	&60.5532	&58.114	&59.0799	&59.55386	&4.148031978\\ \hline
100000	&114.8989	&110.1903	&113.0493	&112.5492	&117.7412	&113.68578	&2.821054449\\ \hline
200000	&234.386	    & 229.6066	&230.8164	&227.1946	&240.2287	&232.44646	&5.065298824\\ \hline
300000	&353.5663	&349.7128	&344.6946	&337.0085	&356.1979	&348.23602	&7.625338651\\ \hline
400000	&478.9285	&463.8541	&474.8911	&481.0437	&477.0424	&475.15196	&6.713015465\\ \hline
500000	&589.1112	&580.0919	&593.2032	&591.8553	&705.3167	&611.91566	&52.46218536\\ \hline
        \end{tabular}
    \end{table} 
    
    
    
        \begin{table}[]
        \begin{tabular}{|c|c|c|c|c|c|c|c| }
            \hline
            \multicolumn{8}{|c|}{Algoritmo: Merge Sort} \\ \hline
            \multicolumn{4}{|c|}{} & \multicolumn{4}{c|}{Lenguaje:GO} \\ \hline
              N de Datos &     t1    &  t2         &  t3          &   t4        &    t5     &   Promedio(t)       & desv. s. \\ \hline
            100&	0.5074&	0	&0	&0	&0	&0.10148	&0.226916178\\ \hline
1000&	0.5219	&0.5233	&0.5287	&0.5202	&0.54	&0.52682	&0.00802602\\ \hline
2000	&1.0521	&1.0386	&1.0428	&0.5246	&0.5372	&0.83906	&0.281387985\\ \hline
3000	&2.2709	&1.558	&2.2751	&1.0488	&1.6232	&1.7552	&0.522387619\\ \hline
4000	&2.6873	&2.9265	&3.3548	&3.0641	&2.0366	&2.81386	&0.497009671\\ \hline
5000	&6.1804	&6.8845	&5.8358	&6.5053	&6.9	  & 6.4612	&0.459251277\\ \hline
6000	&2.715	&7.7319	&2.0867	&8.5517	&8.9547	&6.008	&3.329629277\\ \hline
7000	&4.2708	&3.1435	&8.8315	&2.6065	&6.6433&	5.09912	&2.599945692\\ \hline
8000	&3.5225	&3.8262	&2.0935	&8.3747	&4.678	&4.49898	&2.358269166\\ \hline
9000	&4.9988	&7.0219	&7.3279	&7.135	&8.3189&	6.9605	&1.21065276\\ \hline
10000	&3.996	&21.9886	&3.0013	&9.8411	&4.4007	&8.64554	&7.920873498\\ \hline
20000	&11.9906	&17.9899	&9.9957	&13.9928	&11.8163	&13.15706	&3.049887997\\ \hline
30000	&16.9902	&24.9881	&14.9913	&17.8918	&21.8538	&19.34304	&4.023375847\\ \hline
40000	&27.9822	&26.9853	&22.9909	&28.0106	&24.9855	&26.1909	  &2.170490976\\ \hline
50000	&30.9813	&26.6258	&30.8381	&23.6203	&32.9822&	29.00954	&3.799329573\\ \hline
100000	&52.5777	&63.9633	&56.2067	&56.4835	&57.2278	&57.2918&	4.140245704\\ \hline
200000&	96.7328	  &   107.6231	&107.2718	&110.1763	&110.3102	&106.42284	&5.59594188\\ \hline
300000	&153.1654	&161.9075	&147.6206	&273.9708	&170.0507	&181.343	    &52.47938032\\ \hline
400000&	180.8225&	195.6433	&192.2494	&218.5503	&228.3242	&203.11794	&19.65067001\\ \hline
500000	&226.1258&	261.0595	&266.9817	&267.5833	&248.8818	&254.12642	&17.36342565\\ \hline

                \end{tabular}
    \end{table} 
                    \begin{table}[]
        \begin{tabular}{|c|c|c|c|c|c|c|c| }
            \hline
            \multicolumn{8}{|c|}{Algoritmo: Merge Sort} \\ \hline
            \multicolumn{4}{|c|}{} & \multicolumn{4}{c|}{Lenguaje:Python} \\ \hline
              N de Datos &     t1    &  t2         &  t3          &   t4        &    t5     &   Promedio(t)       & desv. s. \\ \hline
100 &	0.00099802	 &0.001014233	 &0 &	0 &	0.000999451 &	0.602340698	 &0.549895942\\ \hline
1000 &	0.005995512 &	0.008973598 &	0.007950544 &	0.006989479 &	0.005000591 &	6.981945038	 &1.565546045\\ \hline
2000 &	0.014994144	 &0.019010305 &	0.025984049 &	0.013988256 &	0.014013529	 &17.59805679	 &5.122967099\\ \hline
3000 &	0.029980659 &	0.021990776 &	0.051969767 &	0.039978981 &	0.019966602 &	32.7773571	 &13.30882467\\ \hline
4000	 &0.030983925 &	0.037979603 &	0.072960138 &	0.062963486 &	0.022008896 &	45.37920952	 &21.66831998\\ \hline
5000 &	0.057966948	 &0.040974855 &	0.122928381 &	0.107444286 &	0.038974047 &	73.6577034	 &39.00787506\\ \hline
6000 &	0.051971197 &	0.045973539 &	0.103938818 &	0.106939316 &	0.036498547 &	69.06428337	 &33.67727169\\ \hline
7000 &	0.160906315 &	0.057966471 &	0.296339989 &	0.111934662 &	0.043976307 &	134.2247486	 &101.7966247\\ \hline
8000 &	0.067960262 &	0.05896616 &	0.099942446 &	0.156917572 &	0.048482656 &	86.45381927 &	43.83627428\\ \hline
9000 &	0.07095933	 &0.067960978 &	0.096944094 &	0.161907196 &	0.068982124 &	93.35074425 &	40.16448861\\ \hline
10000 &	0.099942446 &	0.076957464 &	0.108938456 &	0.181349754 &	0.054946661	 &104.4269562	 &47.85513018\\ \hline
20000 &	0.159907818 &	0.145916224 &	0.155697346 &	0.216875553 &	0.112954617	 &158.2703114	 &37.58327961\\ \hline
30000 &	0.231253624 &	0.224879026 &	0.239089012 &	0.369790792 &	0.183023691	 &249.6072292 &	70.59820845\\ \hline
40000 &	0.392888308 &	0.310826063 &	0.274857521 &	0.376797676 &	0.359117985	 &342.8975105	 &48.91171307\\ \hline
50000 &	0.420777798 &	0.390955448 &	0.371768951 &	0.551683903 &	0.332072973	 &413.4518147	 &83.70759169\\ \hline
100000 &	0.767641544	 &0.713899851 &	0.910475492 &	0.867524147 &	0.750571966 &	802.0226002	 &83.13742566\\ \hline
200000 &	1.803967714 &	1.712458849 &	2.245170116 &	1.832188606 &	1.635478973	 &1845.852852 &	236.3506309\\ \hline
300000 &	3.931644201 &	2.67054534 &	2.755848169 &	2.798952579 &	2.56230402	 &2943.858862	 &559.5422475\\ \hline
400000 &	3.860989094 &	3.510637999 &	3.683504343 &	3.768096685 &	3.613694191	 &3687.384462	 &135.4046296\\ \hline
500000 &	4.537797928 &	4.659104586 &	4.622467756 &	4.709950686 &	4.4588449	 &4597.633171 &99.81627618\\ \hline
            
            
            \end{tabular}
    \end{table} 
    \subsection{QuickSort}
        \paragraph {}
        Quicksort ha sido históricamente el algoritmo genérico de ordenamiento más rápido conocido en la práctica. Es un algoritmo recursivo del tipo “divide y vencerás”, es fácil de implementar, que permite, en promedio, ordenar n elementos en un tiempo proporcional a n log n.
        \paragraph {}
        La idea básica es ordenar una lista siguiendo los pasos siguientes: se escoge un elemento arbitrario de la lista y se forman tres grupos; el primer grupo, tiene los elementos menores a aquel que se escogió; el segundo, los elementos iguales que el escogido; y el tercero, tiene los elementos más grandes. De forma recursiva, se ordenan el primer y el tercer grupo; luego, se concatenan todos los grupos.
        \paragraph {}
        Este método fue creado por el científico británico Charles Antony Richard Hoare, también conocido como Tony Hoare en 1960, su algoritmo Quicksort es el algoritmo de ordenamiento más ampliamente utilizado en el mundo.
           \begin{figure}[h!]
            \centering
            \includegraphics[width=12cm]{img/QuickSort_1.png}
            \caption{Estrategia que sigue algoritmo para ordenar una secuencia S de n elementos}
            \label{fig:mergesort}
        \end {figure}
    
    
    \begin{table}[]
        \begin{tabular}{|c|c|c|c|c|c|c|c| }
            \hline
            \multicolumn{8}{|c|}{Algoritmo: Quick Sort} \\ \hline
            \multicolumn{4}{|c|}{} & \multicolumn{4}{c|}{Lenguaje: C++} \\ \hline
              N de Datos &     t1    &  t2         &  t3          &   t4        &    t5     &   Promedio(t)       & desv. s. \\ \hline        
100	    &0.0187	&0.0233	&0.0163	&0.0245	&0.0225	&0.02106	&0.003433366 \\ \hline   
1000	&0.2602	&0.3473	&0.4579	&0.2286	&0.259	&0.3106	&0.09350254 \\ \hline   
2000	&0.733	&0.5807	&0.7085	&0.7691	&0.6711	&0.69248	&0.071973968 \\ \hline   
3000	&0.7919	&0.7924	&0.9621	&1.0846	&1.1932	&0.96484	&0.177582412 \\ \hline   
4000	&1.2483	&1.2782	&1.2328	&1.7202	&1.4819	&1.39228	&0.209010543 \\ \hline   
5000	&2.0514	&1.7369	&2.3129	&1.6202	&2.1261	&1.9695	&0.285161349 \\ \hline   
6000	&2.8691	&2.0011	&2.6279	&2.4449	&2.6059	&2.50978	&0.322206071 \\ \hline   
7000	&3.4611	&2.5416	&2.9049	&3.0544	&2.5433	&2.90106	&0.385479056 \\ \hline   
8000	&3.3108	&3.265	&3.8929	&2.9292	&3.9379	&3.46716	&0.435192133 \\ \hline   
9000	&3.8874	&3.6595	&3.986	&3.4005	&3.9895	&3.78458	&0.253128736 \\ \hline   
10000	&4.784	&4.0452	&3.7409	&6.5858	&3.8669	&4.60456	&1.179029806 \\ \hline   
20000	&12.7512	&10.055	&8.812	&11.8333	&8.1281	&10.31592&	1.958918178 \\ \hline   
30000	&12.8663	&12.6601	&23.3719	&15.1841	&13.4952	&15.51552&	4.502389055 \\ \hline   
40000	&18.1782	&16.4566	&20.842	&20.9858	&21.1016	&19.51284&	2.096562822 \\ \hline   
50000	&32.1925	&21.2544	&38.008	&24.691	&35.0562	&30.24042&	7.05128127 \\ \hline   
100000	&60.913	&49.2804	&48.2738	&46.4789	&53.1548	&51.62018&	5.740582542 \\ \hline   
200000	&126.1002	&98.1128	    &106.289	    &101.0139	&109.0098&	108.10514&	10.93242313 \\ \hline   
300000	&189.3729	&165.117	    &153.6852	&155.4696	&160.9269&	164.91432&	14.4001615 \\ \hline   
400000	&255.2373	&213.9224	&225.0151	&215.6328	&229.8306&	227.92764&	16.62251119 \\ \hline   
500000	&306.2959	&275.6362	&280.8843	&283.1553	&308.869	 &    290.96814&	15.43675563 \\ \hline   
        
        
        \end{tabular}
    \end{table}        
        \begin{table}[]
        \begin{tabular}{|c|c|c|c|c|c|c|c| }
            \hline
            \multicolumn{8}{|c|}{Algoritmo: Quick Sort} \\ \hline
            \multicolumn{4}{|c|}{} & \multicolumn{4}{c|}{Lenguaje: GO} \\ \hline
              N de Datos &     t1    &  t2         &  t3          &   t4        &    t5     &   Promedio(t)       & desv. s. \\ \hline  
                100	&0.0156&0	&0	&0	&0	&0.00312	&0.006976532\\ \hline
                1000	&0	&0	&0.1399	&0.2501	&0.3944	&0.15688	&0.169275447\\ \hline
                2000	&0.5932	&0	&0.5816	&0.523	&0.5053	&0.44062	&0.249134486\\ \hline
                3000	&0.5216	&0.7014	&0.5219	&0.5225	&0.525	&0.55848	&0.079905926\\ \hline
                4000	&0.5286	&0.5174	&1.3393	&1.0046	&2.0038	&1.07874	&0.621868329\\ \hline
                5000	&1.08	&1.0216	&1.0381	&1.5512	&1.0355	&1.14528	&0.227962973\\ \hline
                6000	&1.0445	&1.5569	&1.5466	&1.6007	&2.0228	&1.5543	&0.346989661\\ \hline
                7000	&2.6622	&1.0502	&5.1495	&2.1433	&1.0329	&2.40762	&1.687085216\\ \hline
                8000	&2.871	&1.5809	&2.1138	&1.0314	&1.5505	&1.82952	&0.696806162\\ \hline
                9000	&1.5823	&2.3895	&1.5512	&1.5473	&2.4434	&1.90274	&0.469533889\\ \hline
                10000	&2.4711	&2.976	&1.5074	&2.9959	&8.6499	&3.72006	&2.821219629\\ \hline
                20000	&6.259	&3.9964	&5.9965	&4.9961	&5.8911	&5.42782	&0.93062222\\ \hline
                30000	&9.3379	&6.996	&9.0586	&8.5519	&10.994	&8.98768	&1.441327259\\ \hline
                40000	&9.9578	&8.9915	&17.1128	&14.9916	&18.2064	&13.85202	&4.173774013\\ \hline
                50000	&17.2317	&18.9895	&26.8295	&19.5135	&21.4947	&20.81178	&3.691291843\\ \hline
                100000	&44.588	&30.9826	&32.9784	&49.9699	&67.3861	&45.181	&14.72115559\\ \hline
                200000	&41.5332	     &159.2225	&100.7466	&114.9337	&138.887	     &111.0646	&44.85898995\\ \hline
                300000	&79.1607	     &150	    &109.679	    &108.9505	&96.2503	     &108.8081	&26.1448754\\ \hline
                400000	&127.1079	&135.2571	&137.8239	&158.002	      &126.1373	&136.86564	&12.85070179\\ \hline
                500000	&167.5356	&171.6628	&189.0014	&183.3456	&164.9439	&175.29786	&10.40701635\\ \hline
        \end{tabular}
    \end{table} 
    
    
            \begin{table}[]
        \begin{tabular}{|c|c|c|c|c|c|c|c| }
            \hline
            \multicolumn{8}{|c|}{Algoritmo: Quick Sort} \\ \hline
            \multicolumn{4}{|c|}{} & \multicolumn{4}{c|}{Lenguaje: Python} \\ \hline
              N de Datos &     t1    &  t2         &  t3          &   t4        &    t5     &   Promedio(t)       & desv. s. \\ \hline 
100	&0	&0	&0.000998735&	0.000998974&	0	&0.399541855	&0.547095223\\ \hline
1000	&0.009994268	&0.011993647	&0.016989946	&0.012970448&	0.005999088	&11.58947945	&4.032140823\\ \hline
2000	&0.029981136	&0.023008108	&0.064962626	&0.033980846	&0.052969933	&40.98052979	&17.40601087\\ \hline
3000	&0.101940393	&0.039955139	&0.060964823	&0.059965849	&0.068959475	&66.35713577	&22.58279991\\ \hline
4000	&0.108939171	&0.042973995	&0.136921167	&0.138915539	&0.125930548	&110.736084	&39.70372012\\ \hline
5000	&0.09194541	&0.051973581	&0.160906792	&.116374493	&0.15090847	&114.4217491	&44.44353917\\ \hline
6000	&0.08195281	&0.07095933	&0.15591073	&0.166904211	&0.185434818	&132.2323799	&52.14336347\\ \hline
7000	&0.102569818	&0.077952385	&0.186402321	&0.167904139	&0.202881813	&147.5420952	&54.43447383\\ \hline
8000	&0.09894228	&0.09294796	&0.15191102	&0.222416639	&0.130926132	&139.4288063	&52.23674347\\ \hline
9000	&0.117349625	&0.10094142	&0.207879543	&0.13805747	&0.220872641	&157.0201397	&54.18019991\\ \hline
10000	&0.225871801	&0.102969408	&0.158908844	&0.216874361	&0.165904284	&174.1057396	&49.66756315\\ \hline
20000	&0.241859674	&0.246831179	&0.359793901	&0.27782321	&0.340345383	&293.3306694	&54.03631127\\ \hline
30000	&0.406769276	&0.36578846	&0.42777729	&0.49301672	&0.394275188	&417.5253868	&47.79387554\\ \hline
40000	&0.529407501	&0.572183132	&0.62464118	&0.566672802	&0.664618015	&591.5045261	&53.12624451\\ \hline
50000	&0.6591959	&0.721586943	&0.773604155	&0.776862621	&0.686623096	&723.574543	&52.09738708\\ \hline
100000	&1.371208668	&1.430203676	&1.544603109	&1.464600325	&1.51870966	&1465.865088	&69.32784174\\ \hline
200000	&3.062305212	&3.088241339	&3.135392666	&3.151999712	&2.980718374	&3083.731461	&67.86116961\\ \hline
300000	&4.930557966	&4.727021456	&4.675037146	&4.993543625	&4.923748016	&4849.981642	&139.882852\\ \hline
400000	&7.751039982	&6.572371244	&6.236221552	&6.337382078	&6.679156542	&6715.23428	&605.5638658\\ \hline
500000	&8.079737902	&9.067592859	&8.209435701	&8.917642117	&8.596114874	&8574.104691	&429.946211\\ \hline
              
             \end{tabular}
    \end{table}          
              
    \subsection{HeapSort}
            \begin{figure}[h!]
            \centering
            \includegraphics[width=12cm]{img/HeapSort_1.png}
            \caption{Estrategia que sigue algoritmo para ordenar una secuencia S de n elementos}
            \label{fig:mergesort}
        \end {figure} 
        
            \begin{table}[]
        \begin{tabular}{|c|c|c|c|c|c|c|c| }
            \hline
            \multicolumn{8}{|c|}{Algoritmo: Heap Sort} \\ \hline
            \multicolumn{4}{|c|}{} & \multicolumn{4}{c|}{Lenguaje: C++} \\ \hline
              N de Datos &     t1    &  t2         &  t3          &   t4        &    t5     &   Promedio(t)       & desv. s. \\ \hline
                 100	&0.0187	&0.0233	&0.0163	&0.0245	&0.0225	&0.02106	&0.003433366\\ \hline
                1000	&0.2602	&0.3473	&0.4579	&0.2286	&0.259	&0.3106	&0.09350254\\ \hline
                2000	&0.733	&0.5807	&0.7085	&0.7691	&0.6711	&0.69248	&0.071973968\\ \hline
                3000	&0.7919	&0.7924	&0.9621	&1.0846	&1.1932	&0.96484	&0.177582412\\ \hline
                4000	&1.2483	&1.2782	&1.2328	&1.7202	&1.4819	&1.39228	&0.209010543\\ \hline
                5000	&2.0514	&1.7369	&2.3129	&1.6202	&2.1261	&1.9695	&0.285161349\\ \hline
                6000	&2.8691	&2.0011	&2.6279	&2.4449	&2.6059	&2.50978	&0.322206071\\ \hline
                7000	&3.4611	&2.5416	&2.9049	&3.0544	&2.5433	&2.90106	&0.385479056\\ \hline
                8000	&3.3108	&3.265	&3.8929	&2.9292	&3.9379	&3.46716	&0.435192133\\ \hline
                9000	&3.8874	&3.6595	&3.986	&3.4005	&3.9895	&3.78458	&0.253128736\\ \hline
                10000	&4.784	&4.0452	&3.7409	&6.5858	&3.8669	&4.60456	&1.179029806\\ \hline
                20000	&12.7512	&10.055	&8.812	&11.8333	&8.1281	&10.31592&	1.958918178\\ \hline
                30000	&12.8663	&12.6601	&23.3719	&15.1841	&13.4952	&15.51552&	4.502389055\\ \hline
                40000	&18.1782	&16.4566	&20.842	&20.9858	&21.1016	&19.51284&	2.096562822\\ \hline
                50000	&32.1925	&21.2544	&38.008	&24.691	&35.0562	&30.24042&	7.05128127\\ \hline
                100000	&60.913	&49.2804	&48.2738	&46.4789	&53.1548	&51.62018&	5.740582542\\ \hline
                200000	&126.1002	&98.1128	&106.289	&101.0139	&109.0098&	108.10514&	10.93242313\\ \hline
                300000	&189.3729	&165.117	&153.6852	&155.4696	&160.9269&	164.91432&	14.4001615\\ \hline
                400000	&255.2373	&213.9224	&225.0151&	215.6328	&229.8306	&227.92764&	16.62251119\\ \hline
                500000	&306.2959	&275.6362	&280.8843&	283.1553	&308.869	&290.96814&	15.43675563\\ \hline

    \end{tabular}
    \end{table}
    
             \begin{table}[]
        \begin{tabular}{|c|c|c|c|c|c|c|c| }
            \hline
            \multicolumn{8}{|c|}{Algoritmo: Heap Sort} \\ \hline
            \multicolumn{4}{|c|}{} & \multicolumn{4}{c|}{Lenguaje: GO} \\ \hline
              N de Datos &     t1    &  t2         &  t3          &   t4        &    t5     &   Promedio(t)       & desv. s. \\ \hline       
                100	&0.0156	&0	&0	&0	&0	&0.00312	&0.006976532\\ \hline
                1000	&0	&0	&0.1399	&0.2501	&0.3944	&0.15688	&0.169275447\\ \hline
                2000	&0.5932	&0	&0.5816	&0.523	&0.5053	&0.44062	&0.249134486\\ \hline
                3000	&0.5216	&0.7014	&0.5219	&0.5225	&0.525	&0.55848	&0.079905926\\ \hline
                4000	&0.5286	&0.5174	&1.3393	&1.0046	&2.0038	&1.07874	&0.621868329\\ \hline
                5000	&1.08	&1.0216	&1.0381	&1.5512	&1.0355	&1.14528	&0.227962973\\ \hline
                6000	&1.0445	&1.5569	&1.5466	&1.6007	&2.0228	&1.5543	&0.346989661\\ \hline
                7000	&2.6622	&1.0502	&5.1495	&2.1433	&1.0329	&2.40762	&1.687085216\\ \hline
                8000	&2.871	&1.5809	&2.1138	&1.0314	&1.5505	&1.82952	&0.696806162\\ \hline
                9000	&1.5823	&2.3895	&1.5512	&1.5473	&2.4434	&1.90274	&0.469533889\\ \hline
                10000	&2.4711	&2.976	&1.5074	&2.9959	&8.6499	&3.72006	&2.821219629\\ \hline
                20000	&6.259	&3.9964	&5.9965	&4.9961	&5.8911	&5.42782	&0.93062222\\ \hline
                30000	&9.3379	&6.996	&9.0586	&8.5519	&10.994	&8.98768	&1.441327259\\ \hline
                40000	&9.9578	&8.9915	&17.1128	&14.9916	&18.2064	&13.85202&	4.173774013\\ \hline
                50000	&17.2317	&18.9895	&26.8295	&19.5135	&21.4947	&20.81178&	3.691291843\\ \hline
                100000	&44.588	&30.9826	&32.9784	&49.9699	&67.3861	&45.181	&14.72115559\\ \hline
                200000	&41.5332	&159.2225	&100.7466	&114.9337	&138.887	&111.0646	&44.85898995\\ \hline
                300000	&79.1607	&150	&109.679	&108.9505	&96.2503	&108.8081	&26.1448754\\ \hline
                400000	&127.1079	&135.2571&	137.8239	&158.002&	126.1373&	136.86564	&12.85070179\\ \hline
                500000	&167.5356	&171.6628&	189.0014	&183.3456	&164.9439	&175.29786&	10.40701635\\ \hline        
        \end{tabular}
    \end{table}
    
    
    
    \begin{table}[]
        \begin{tabular}{|c|c|c|c|c|c|c|c| }
            \hline
            \multicolumn{8}{|c|}{Algoritmo: Heap Sort} \\ \hline
            \multicolumn{4}{|c|}{} & \multicolumn{4}{c|}{Lenguaje: Python} \\ \hline
              N de Datos &     t1    &  t2         &  t3          &   t4        &    t5     &   Promedio(t)       & desv. s. \\ \hline   
    100	    &0	            &0	            &0.000998735	&0.000998974	&0	             &0.399541855	 &0.547095223\\ \hline
    1000	&0.009994268	&0.011993647	&0.016989946	&0.012970448	&0.005999088	&11.58947945	 	&4.032140823\\ \hline
    2000	&0.029981136	&0.023008108	&0.064962626	&0.033980846	&0.052969933	&40.98052979	 	&17.40601087\\ \hline
    3000	&0.101940393	&0.039955139	&0.060964823	&0.059965849	&0.068959475	&66.35713577	 	&22.58279991\\ \hline
    4000	&0.108939171	&0.042973995	&0.136921167	&0.138915539	&0.125930548	&110.736084	        &39.70372012\\ \hline
    5000	&0.09194541	    &0.051973581	&0.160906792	&0.116374493	&0.15090847	    &114.4217491	 	&44.44353917\\ \hline
    6000	&0.08195281	    &0.07095933	    &0.15591073	    &0.166904211	&0.185434818	&132.2323799	 	&52.14336347\\ \hline
    7000	&0.102569818	&0.077952385	&0.186402321	&0.167904139	&0.202881813	&147.5420952	 	&54.43447383\\ \hline
    8000	&0.09894228	    &0.09294796	    &0.15191102	    &0.222416639	&0.130926132	&139.4288063	 	&52.23674347\\ \hline
    9000	&0.117349625	&0.10094142	    &0.207879543	&0.13805747	    &0.220872641	&157.0201397	 	&54.18019991\\ \hline
    10000	&0.225871801	&0.102969408	&0.158908844	&0.216874361	&0.165904284	&174.1057396	 	&49.66756315\\ \hline
    20000	&0.241859674	&0.246831179	&0.359793901	&0.27782321	    &0.340345383	&293.3306694	 	&54.03631127\\ \hline
    30000	&0.406769276	&0.36578846	    &0.42777729	    &0.49301672	    &0.394275188	&417.5253868	 	&47.79387554\\ \hline
    40000	&0.529407501	&0.572183132	&0.62464118	    &0.566672802	&0.664618015	&591.5045261	 	&53.12624451\\ \hline
    50000	&0.6591959	    &0.721586943	&0.773604155	&0.776862621	&0.686623096	&723.574543	 	&52.09738708\\ \hline
    100000	&1.371208668	&1.430203676	&1.544603109	&1.464600325	&1.51870966	     &1465.865088	 	&69.32784174\\ \hline
    200000	&3.062305212	&3.088241339	&3.135392666	&3.151999712	&2.980718374	&3083.731461	 	&67.86116961\\ \hline
    300000	&4.930557966	&4.727021456	&4.675037146	&4.993543625	&4.923748016	&4849.981642	 	&139.882852\\ \hline
    400000	&7.751039982	&6.572371244	&6.236221552	&6.337382078	&6.679156542	&6715.23428	 	&605.5638658\\ \hline
    500000	&8.079737902	&9.067592859	&8.209435701	&8.917642117	&8.596114874	&8574.104691	 	&429.946211\\ \hline
        \end{tabular}
    \end{table}
              
    \subsection{TreeSort}
    La clasificación de árbol es un algoritmo de clasificación que se basa en la estructura de datos del árbol de búsqueda binaria. Primero crea un árbol de búsqueda binario a partir de los elementos de la lista o matriz de entrada y luego realiza un recorrido en orden en el árbol de búsqueda binario creado para ordenar los elementos.
        \subsubsection{Costo Computacional}
        \subsubsection{Resultado de las pruebas}
        \begin{figure}[h!]
            \centering
            \includegraphics[width=12cm]{img/treeSort_1.png}
            \caption{Estrategia que sigue algoritmo para ordenar una secuencia S de n elementos}
            \label{fig:mergesort}
        \end {figure} 
    \begin{table}[]
        \begin{tabular}{|c|c|c|c|c|c|c|c| }
            \hline
            \multicolumn{8}{|c|}{Algoritmo: Tree Sort} \\ \hline
            \multicolumn{4}{|c|}{} & \multicolumn{4}{c|}{Lenguaje: C++} \\ \hline
              N de Datos &     t1    &  t2         &  t3          &   t4        &    t5     &   Promedio(t)       & desv. s. \\ \hline
            100   & 0.0345  &   0.0215  &   0.0282   &  0.0198   & 0.0248  & 0.02576  & 0.005850897 \\ \hline
            1000   & 0.2949  &	0.2878  &	0.2575   &	0.2509   & 0.3067  & 0.27956  & 0.024227216 \\ \hline
            2000   & 0.5418  &	0.6543  &	0.4691   &	0.8348   & 0.4672  & 0.59344  & 0.154937287 \\ \hline
            3000   & 1.2841  &	1.3463  &	0.7256   &	0.761    & 1.2948  & 1.08236  & 0.310662217 \\ \hline
            4000   & 1.4992  &	0.9804  &	1.2329   &	2.0372   & 1.1113  & 1.3722   & 0.418131122 \\ \hline
            5000   & 3.241   &   1.7043  &	1.6303   &	2.3498   & 1.629   & 2.11088  & 0.70048675  \\ \hline
            6000   & 2.3611  &	3.3129  &	4.3816	 &  2.7473   & 3.162   & 3.19298  & 0.761382471 \\ \hline
            7000   & 2.6891  &	3.2629  &	2.348	 &  2.3321   & 3.9669  & 2.9198   & 0.69636647  \\ \hline
            8000   & 7.2664  &	3.7294  &	8.6777   &	4.3366   & 5.2377  & 5.84956  & 2.071491642 \\ \hline
            9000   & 3.2077  &	3.1763  &	4.7729   &	3.4111   & 3.5677  & 3.62714  & 0.659954353 \\ \hline
            10000   & 3.5995  &	3.7944  &	3.8475   &	3.9555   & 7.1328  & 4.46594  & 1.496399306 \\ \hline
            20000   & 8.9341  &	12.2247 &	22.0834  &  9.514    & 9.0016  & 12.35156 & 5.605305854 \\ \hline
            30000   & 20.4284 &	15.7075 &   14.5408  &  16.3092  & 15.609  & 16.51898 & 2.276363981 \\ \hline
            40000   & 24.0826	&   23.9909 &	33.5676  &  22.5173  & 26.759  & 26.18348 & 4.402227591 \\ \hline
            50000   & 122.234	&   42.6545 &	44.3492  &  37.6127  & 39.9763 & 57.36534 & 36.35353237 \\ \hline
            100000  & 70.2404	&   77.7421 &	78.2011  &  70.5764  & 77.8051 & 74.91302 & 4.117612814 \\ \hline
            200000   & 174.7876&	193.7067&	172.2818 &	172.4993 & 180.8891& 178.8329 & 9.011849659 \\ \hline
            300000   & 286.8869&	287.1223&	310.3984 &	279.5616 & 280.2003& 288.8339 & 12.57241663 \\ \hline
            400000   & 404.6286&	437.1003&	426.6823 &	414.3129 & 413.0981& 419.16444& 12.74599799 \\ \hline
            500000   & 545.8179&	552.81  &	560.67   &	551.5422 & 565.9059& 555.3492 & 7.930026684 \\ \hline
            \end{tabular}
    \end{table}
	\begin{table}[]
        \begin{tabular}{|c|c|c|c|c|c|c|c| }
            \hline
            \multicolumn{8}{|c|}{Algoritmo: Tree Sort} \\ \hline
            \multicolumn{4}{|c|}{} & \multicolumn{4}{c|}{Lenguaje: GO} \\ \hline
              N de Datos &     t1    &  t2         &  t3          &   t4        &    t5     &   Promedio(t)       & desv. s. \\ \hline
100	    &0	    &0	    &0	        &0	    &0	    &0	      &0              \\ \hline
1000	&0	    &0	    &0	        & 0.5229&0	    &0.10458  &0.233847989     \\ \hline
2000	&1.0602	&0	    & 0.5173	&0.5213	&0.6305	&0.54586  &0.377853004\\ \hline
3000	&0.5163	&0.5263	& 0.5168	&1.0277	&0.4843	&0.61428  &0.231653733\\ \hline
4000	&1.4224	&2.1089	& 1.9989	&1.9995	&1.5577	&1.81748  &0.305999987\\ \hline
5000	&1.5459	&1.4908	& 0.9963	&1.7008	&1.8522	&1.5172	  &0.323570479\\ \hline
6000	&2.5209	&2.6206	& 2.9977	&2.2495	&1.5574	&2.38922  &0.536789164\\ \hline
7000	&3.6788	&9.0697	& 4	2.9965	&1.6009	&4.26918&4.26918    &2.837491464\\ \hline
8000	&2.0718	&2.0822	& 4.7402	&5.9971	&6.4824	&4.27474	&2.104585229\\ \hline
9000	&2.0692	&2.8883	& 6.0346	&4.5788	&4.0877	&3.93172	&1.534833267\\ \hline
10000	&8.5031	&7.0278	& 7.0258	&8.9303	&3.0253	&6.90246	&2.33117331\\ \hline
20000	&16.9883&	9.9934	& 11.4141	&11.2215	&10.4236	&12.00818	&2.843572749\\ \hline
30000	&20.9716&	21.9881	& 16.9907	&19.2206	&15.989	    &19.032	    &2.547690131\\ \hline
40000	&17.9895&	21.9853	& 20.9874	&21.9866	&22.9862	&21.187	    &1.922082861\\ \hline
50000	&33.9807&	29.9834	& 28.9829	&31.979	    &26.983	    &30.3818	&2.700491356\\ \hline
100000	&60.0404&	61.9663	& 63.9625	&53.97	    &72.955	    &62.57884	&6.901271991\\ \hline
200000	&116.2649&	130.0599&	145.3944 &140.5367	&129.0323	&132.25764	&11.31501115\\ \hline
300000	&214.4081&	218.5863&	282.0565 &201.3208	&253.0039	&233.87512	&33.01458859\\ \hline
400000	&305.4279&	307.92	& 367.4944	 &341.7326	&384.1765	&341.35028	&35.09183418\\ \hline
500000	&419.8142&	429.1118&	493.0747 &515.5093	&442.0059	&459.90318  & 42.03551579\\ \hline

    
       \end{tabular}
   \end{table}
	%\clearpage
	%\bibliographystyle{apalike}
	%\bibliographystyle{IEEEtranN}
	%\bibliography{bibliography}
		

	\begin{table}[]
        \begin{tabular}{|c|c|c|c|c|c|c|c| }
            \hline
            \multicolumn{8}{|c|}{Algoritmo: Tree Sort} \\ \hline
            \multicolumn{4}{|c|}{} & \multicolumn{4}{c|}{Lenguaje: Python} \\ \hline
              N de Datos &     t1    &  t2         &  t3          &   t4        &    t5     &   Promedio(t)       & desv. s. \\ \hline
100	    &0.015624285	&0	        &0.001001835	&0.000999928	&0	        &3.525209427	&6.782077381 \\ \hline
1000	&0.008558035	&0.003997564	&0.004995108	&0.004993439	&0.010315895	&6.572008133	&2.718788759\\ \hline
2000	&0.019988537	&0.026985168	&0.028544664	&0.011993408	&0.021991014	&21.90055847	&6.553873617\\ \hline
3000	&0.024986029	&0.042976379	&0.049971104	&0.049968719	&0.079953671	&49.57118034	&19.82003258\\ \hline
4000	&0.040974855	&0.091937304	&0.045972109	&0.070964336	&0.067958117	&63.56134415	&20.60862468\\ \hline
5000	&0.044970989	&0.11288166	&0.042973995	&0.101940393	&0.09394598	&79.34260368	&32.98814407\\ \hline
6000	&0.070955515	&0.071353912	&0.058964968	&0.099941492	&0.108936548	&82.03048706	&21.29203294\\ \hline
7000	&0.091950655	&0.099225044	&0.135372639	&0.132360697	&0.10193944	&112.1696949	&20.16856982\\ \hline
8000	&0.120291233	&0.094572067	&0.080644131	&0.139489412	&0.112987995	&109.5969677	&22.82235213\\ \hline
9000	&0.113935947	&0.173901558	&0.073012352	&0.108042202	&0.212877989	&136.3540096	&56.08482712\\ \hline
10000	&0.132925034	&0.090055466	&0.129489422	&0.1939466	&0.133127928	&135.9088898	&37.17781267\\ \hline
20000	&0.315265894	&0.239135265	&0.208152771	&0.246989012	&0.255201578	&252.948904	&39.12004677\\ \hline
30000	&0.378284216	&0.381990671	&0.358812094	&0.447550774	&0.349073887	&383.1423283	&38.49024037\\ \hline
40000	&0.55368185	&0.62367034	&0.560783863	&0.627627611	&0.533614874	&579.8757076	&42.97934242\\ \hline
50000	&0.716868401	&0.73401022	&0.728899479	&0.708563566	&0.729392529	&723.5468388	&10.50482242\\ \hline
100000	&1.756626368	&1.585929394	&1.614357233	&1.669086695	&1.468574524	&1618.914843	&106.292876\\ \hline
200000	&3.741744757	&3.7024014	&3.825145483	&3.708697319	&5.033451557	&4002.288103	&578.508426\\ \hline
300000	&6.054645538	&5.865937948	&6.147264957	&5.924173594	&6.492034435	&6096.811295	&246.7950222\\ \hline
400000	&8.224924326	&8.243051291	&8.770008802	&8.109434605	&9.380488396	&8545.581484	&531.9790418\\ \hline
500000	&9.972488165	&11.47141552	&10.28874731	&10.39334488	&12.78968287	&10983.13575	&1156.877989\\ \hline

       \end{tabular}
   \end{table}
\end{document} 