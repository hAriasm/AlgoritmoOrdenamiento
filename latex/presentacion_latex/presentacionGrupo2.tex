%%%%%%%%%%%%%%%%%%%%%%%%%%%%%%%%%%%%%%%%%
% Beamer Presentation
% LaTeX Template
% Version 2.0 (March 8, 2022)
%
% This template originates from:
% https://www.LaTeXTemplates.com
%
% Author:
% Vel (vel@latextemplates.com)
%
% License:
% CC BY-NC-SA 4.0 (https://creativecommons.org/licenses/by-nc-sa/4.0/)
%
%%%%%%%%%%%%%%%%%%%%%%%%%%%%%%%%%%%%%%%%%

%----------------------------------------------------------------------------------------
%	PACKAGES AND OTHER DOCUMENT CONFIGURATIONS
%----------------------------------------------------------------------------------------

\documentclass[
	11pt, % Set the default font size, options include: 8pt, 9pt, 10pt, 11pt, 12pt, 14pt, 17pt, 20pt
	%t, % Uncomment to vertically align all slide content to the top of the slide, rather than the default centered
	%aspectratio=169, % Uncomment to set the aspect ratio to a 16:9 ratio which matches the aspect ratio of 1080p and 4K screens and projectors
]{beamer}

\graphicspath{{Images/}{./}} % Specifies where to look for included images (trailing slash required)

\usepackage{booktabs} % Allows the use of \toprule, \midrule and \bottomrule for better rules in tables

%----------------------------------------------------------------------------------------
%	SELECT LAYOUT THEME
%----------------------------------------------------------------------------------------

% Beamer comes with a number of default layout themes which change the colors and layouts of slides. Below is a list of all themes available, uncomment each in turn to see what they look like.

%\usetheme{default}
%\usetheme{AnnArbor}
%\usetheme{Antibes}
%\usetheme{Bergen}
%\usetheme{Berkeley}
%\usetheme{Berlin}
%\usetheme{Boadilla}
%\usetheme{CambridgeUS}
%\usetheme{Copenhagen}
%\usetheme{Darmstadt}
%\usetheme{Dresden}
%\usetheme{Frankfurt}
%\usetheme{Goettingen}
%\usetheme{Hannover}
%\usetheme{Ilmenau}
%\usetheme{JuanLesPins}
%\usetheme{Luebeck}
\usetheme{Madrid}
%\usetheme{Malmoe}
%\usetheme{Marburg}
%\usetheme{Montpellier}
%\usetheme{PaloAlto}
%\usetheme{Pittsburgh}
%\usetheme{Rochester}
%\usetheme{Singapore}
%\usetheme{Szeged}
%\usetheme{Warsaw}

%----------------------------------------------------------------------------------------
%	SELECT COLOR THEME
%----------------------------------------------------------------------------------------

% Beamer comes with a number of color themes that can be applied to any layout theme to change its colors. Uncomment each of these in turn to see how they change the colors of your selected layout theme.

%\usecolortheme{albatross}
%\usecolortheme{beaver}
%\usecolortheme{beetle}
%\usecolortheme{crane}
%\usecolortheme{dolphin}
%\usecolortheme{dove}
%\usecolortheme{fly}
%\usecolortheme{lily}
%\usecolortheme{monarca}
%\usecolortheme{seagull}
%\usecolortheme{seahorse}
%\usecolortheme{spruce}
%\usecolortheme{whale}
%\usecolortheme{wolverine}

%----------------------------------------------------------------------------------------
%	SELECT FONT THEME & FONTS
%----------------------------------------------------------------------------------------

% Beamer comes with several font themes to easily change the fonts used in various parts of the presentation. Review the comments beside each one to decide if you would like to use it. Note that additional options can be specified for several of these font themes, consult the beamer documentation for more information.

\usefonttheme{default} % Typeset using the default sans serif font
%\usefonttheme{serif} % Typeset using the default serif font (make sure a sans font isn't being set as the default font if you use this option!)
%\usefonttheme{structurebold} % Typeset important structure text (titles, headlines, footlines, sidebar, etc) in bold
%\usefonttheme{structureitalicserif} % Typeset important structure text (titles, headlines, footlines, sidebar, etc) in italic serif
%\usefonttheme{structuresmallcapsserif} % Typeset important structure text (titles, headlines, footlines, sidebar, etc) in small caps serif

%------------------------------------------------

%\usepackage{mathptmx} % Use the Times font for serif text
\usepackage{palatino} % Use the Palatino font for serif text

%\usepackage{helvet} % Use the Helvetica font for sans serif text
\usepackage[default]{opensans} % Use the Open Sans font for sans serif text
%\usepackage[default]{FiraSans} % Use the Fira Sans font for sans serif text
%\usepackage[default]{lato} % Use the Lato font for sans serif text

%----------------------------------------------------------------------------------------
%	SELECT INNER THEME
%----------------------------------------------------------------------------------------

% Inner themes change the styling of internal slide elements, for example: bullet points, blocks, bibliography entries, title pages, theorems, etc. Uncomment each theme in turn to see what changes it makes to your presentation.

%\useinnertheme{default}
\useinnertheme{circles}
%\useinnertheme{rectangles}
%\useinnertheme{rounded}
%\useinnertheme{inmargin}

%----------------------------------------------------------------------------------------
%	SELECT OUTER THEME
%----------------------------------------------------------------------------------------

% Outer themes change the overall layout of slides, such as: header and footer lines, sidebars and slide titles. Uncomment each theme in turn to see what changes it makes to your presentation.

%\useoutertheme{default}
%\useoutertheme{infolines}
%\useoutertheme{miniframes}
%\useoutertheme{smoothbars}
%\useoutertheme{sidebar}
%\useoutertheme{split}
%\useoutertheme{shadow}
%\useoutertheme{tree}
%\useoutertheme{smoothtree}

%\setbeamertemplate{footline} % Uncomment this line to remove the footer line in all slides
%\setbeamertemplate{footline}[page number] % Uncomment this line to replace the footer line in all slides with a simple slide count

%\setbeamertemplate{navigation symbols}{} % Uncomment this line to remove the navigation symbols from the bottom of all slides

%----------------------------------------------------------------------------------------
%	PRESENTATION INFORMATION
%----------------------------------------------------------------------------------------

\title[ALG ORD]{Algoritmos de Ordenamiento} % The short title in the optional parameter appears at the bottom of every slide, the full title in the main parameter is only on the title page

\subtitle{Práctica 01} % Presentation subtitle, remove this command if a subtitle isn't required

\author[ ]{EDER   AMPUERO \and HOWARD   ARANZAMENDI \and JOSE EDISON PEREZ \and HENRRY   ARIAS} % Presenter name(s), the optional parameter can contain a shortened version to appear on the bottom of every slide, while the main parameter will appear on the title slide

\institute[UC]{Universidad Nacional San Agustín de Arequipa \\ \smallskip \textit{eampuero@unsa.edu.pe \and
haranzamendi@unsa.edu.pe  \and
jperezma@unsa.edu.pe \and  hariasm@unsa.edu.pe}} % Your institution, the optional parameter can be used for the institution shorthand and will appear on the bottom of every slide after author names, while the required parameter is used on the title slide and can include your email address or additional information on separate lines

\date[\today]{Arequipa \\ \today} % Presentation date or conference/meeting name, the optional parameter can contain a shortened version to appear on the bottom of every slide, while the required parameter value is output to the title slide

%----------------------------------------------------------------------------------------

\begin{document}

%----------------------------------------------------------------------------------------
%	TITLE SLIDE
%----------------------------------------------------------------------------------------

\begin{frame}
	\titlepage % Output the title slide, automatically created using the text entered in the PRESENTATION INFORMATION block above
\end{frame}

%----------------------------------------------------------------------------------------
%	TABLE OF CONTENTS SLIDE
%----------------------------------------------------------------------------------------

% The table of contents outputs the sections and subsections that appear in your presentation, specified with the standard \section and \subsection commands. You may either display all sections and subsections on one slide with \tableofcontents, or display each section at a time on subsequent slides with \tableofcontents[pausesections]. The latter is useful if you want to step through each section and mention what you will discuss.

\begin{frame}
	\frametitle{Tabla de contenidos} % Slide title, remove this command for no title
	
	\tableofcontents % Output the table of contents (all sections on one slide)
	%\tableofcontents[pausesections] % Output the table of contents (break sections up across separate slides)
\end{frame}

%----------------------------------------------------------------------------------------
%	PRESENTATION BODY SLIDES
%----------------------------------------------------------------------------------------

\section{Características} % Sections are added in order to organize your presentation into discrete blocks, all sections and subsections are automatically output to the table of contents as an overview of the talk but NOT output in the presentation as separate slides

%------------------------------------------------

\subsection{Equipo usado para pruebas}

\begin{frame}
	\frametitle{Características del equipo}
	
	Se uso un único equipo con la misma data para todas las pruebas.

		\begin{figure}
		\includegraphics[width=0.8\linewidth]{caracteristicas.png}
		\caption{Características del equipo de pruebas}
	   \end{figure}

\end{frame}

%------------------------------------------------
\section{Algoritmos}
\subsection{QUICK SORT}
\begin{frame}
	\frametitle{QUICK SORT}
    \begin{columns}[t] % The "c" option specifies centered vertical alignment while the "t" option is used for top vertical alignment
		\begin{column}{0.45\textwidth} % Left column width
			
        Quicksort ha sido históricamente el algoritmo genérico de ordenamiento más rápido conocido en la práctica. Es un algoritmo recursivo del tipo "divide y vencerás", y fácil de implementar.

        Costo computacional: Caso promedio tarda N x log N, en el peor caso puede llegar a tardar $N^2$.

		\end{column}		
		\begin{column}{0.5\textwidth} % Right column width
			\begin{figure}
		      \includegraphics[width=0.8\linewidth]{quickSort.png}
		      \caption{Quick Sort}
	   \end{figure}
		
		\end{column}
	\end{columns}
\end{frame}

%------------------------------------------------


\begin{frame}
	\frametitle{Resultados}
	\begin{figure}
		\includegraphics[width=0.8\linewidth]{QuickSort_1.png}
		\caption{Quick Sort}
	\end{figure}
\end{frame}

%------------------------------------------------

\subsection{Merge Sort}

\begin{frame}
	\frametitle{MERGE SORT}
    \begin{columns}[t] % The "c" option specifies centered vertical alignment while the "t" option is used for top vertical alignment
		\begin{column}{0.45\textwidth} % Left column width
			 \begin{itemize} 
              \item Es un algoritmo recursivo bastante eficiente para ordenar un array.
        \item Usa la técnica de divide y vencerás, la cual consiste en dividir el problema en sub problemas del mismo tipo que a su vez se dividirán hasta que sean suficientemente pequeños o triviales
        \item Costo computacional: T(N) = Nlog2 N
    \end{itemize}

		\end{column}		
		\begin{column}{0.5\textwidth} % Right column width
			\begin{figure}
		      \includegraphics[width=0.8\linewidth]{mergesort.png}
		      \caption{Merge Sort}
	   \end{figure}
		
		\end{column}
	\end{columns}
\end{frame}
\begin{frame}
	\frametitle{Resultados}
	
	\begin{figure}
		\includegraphics[width=0.8\linewidth]{mergeSort_1.png}
		\caption{Merge Sort}
	\end{figure}
\end{frame}
%------------------------------------------------

\subsection{Heap Sort}

\begin{frame}
	\frametitle{HEAP SORT}
    \begin{columns}[t] % The "c" option specifies centered vertical alignment while the "t" option is used for top vertical alignment
		\begin{column}{0.45\textwidth} % Left column width
			 HeapsortHeapsort (Williams, 1964)

        En la práctica su coste es superior al de quicksort, ya que el factor constante multiplicativo del término nlogn es mayor.

		\end{column}		
		\begin{column}{0.5\textwidth} % Right column width
			\begin{figure}
		      \includegraphics[width=0.8\linewidth]{mergesort.png}
		      \caption{Merge Sort}
	   \end{figure}
		
		\end{column}
	\end{columns}
\end{frame}

\begin{frame}
	\frametitle{Resultados}
	
	\begin{figure}
		\includegraphics[width=0.8\linewidth]{HeapSort_1.png}
		\caption{Heap Sort}
	\end{figure}
\end{frame}
%------------------------------------------------

\subsection{Tree Sort}

\begin{frame}
	\frametitle{TREE SORT}
    \begin{columns}[t] % The "c" option specifies centered vertical alignment while the "t" option is used for top vertical alignment
		\begin{column}{0.45\textwidth} % Left column width
			 La clasificación de árbol es un algoritmo de clasificación que se basa en la estructura de datos del árbol de búsqueda binaria.
		\end{column}		
		\begin{column}{0.5\textwidth} % Right column width
			\begin{figure}
		      \includegraphics[width=0.8\linewidth]{treesorttwo_1.png}
		      \caption{Tree Sort}
	   \end{figure}
		
		\end{column}
	\end{columns}
\end{frame}

\begin{frame}
	\frametitle{Resultados}
	
	\begin{figure}
		\includegraphics[width=0.8\linewidth]{treeSort_1.png}
		\caption{Tree Sort}
	\end{figure}
\end{frame}
 
%------------------------------------------------

\begin{frame} % Use [allowframebreaks] to allow automatic splitting across slides if the content is too long
	\frametitle{References}
	
	\begin{thebibliography}{99} % Beamer does not support BibTeX so references must be inserted manually as below, you may need to use multiple columns and/or reduce the font size further if you have many references
		\footnotesize % Reduce the font size in the bibliography
		
		\bibitem[Smith, 2022]{p1}
			John Smith (2022)
			\newblock Publication title
			\newblock \emph{Journal Name} 12(3), 45 -- 678.
			
		\bibitem[Kennedy, 2023]{p2}
			Annabelle Kennedy (2023)
			\newblock Publication title
			\newblock \emph{Journal Name} 12(3), 45 -- 678.
	\end{thebibliography}
\end{frame}
 

%----------------------------------------------------------------------------------------
%	CLOSING SLIDE
%----------------------------------------------------------------------------------------

\begin{frame}[plain] % The optional argument 'plain' hides the headline and footline
	\begin{center}
		{\Huge Fin}
		
		\bigskip\bigskip % Vertical whitespace
		
		{\LARGE Preguntas? Comentarios?}
	\end{center}
\end{frame}

%----------------------------------------------------------------------------------------

\end{document} 